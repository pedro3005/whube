\chapter{Digging in deeper to Whube}
In this chapter we will discuss more advanced features of the Whube platform.
\section{\$SITE\_PREFIX}
asd
\section{Kicking ass with \$argv}
Let's take a look at a new request. \texttt{http://whube.com/t/foo/bar/baz}.
Breaking this back up, the request turns into \texttt{controller.php?p=foo/bar/baz}.
If you missed this, go back and review the MVC Basics. \texttt{controller.php} will
strip out the first item ( \texttt{foo} ) to detect the content script. \texttt{content.php}
will take the rest of the URL past a forward slash and put it into an argument list.
\texttt{content.php} sets up \$argv and \$argc just like this:
\begin{verbatim}

    // handling request http://whube.com/t/foo/bar/baz

    $argv = array(   // it's not actually done this way
        'bar',       // but it makes things really clear
        'baz'        // compared to the push back system
    );               // it uses.

    $argc = sizeof( $argv ); // this is actually how it's done.

\end{verbatim}
A simple script using this would look something like:
\begin{verbatim}
<?php
    $TITLE   = 'Argument Listing';
    $CONTENT = "<h1>Arguments passed in:</h1>\n";

    foreach( $argv as $arg ) {
        $CONTENT .= $arg . "<br />\n";
    }

?>
\end{verbatim}
If the foreach loop is new to you, you are missing out! It's really
handy if you want to avoid cookiecutter code like a normal for loop.
\section{Pimping out with useScript()}
\texttt{useScript()} is a pretty kickass function of the whube codebase.
It does lots of really fun things when you wish to manage what scripts should be included.
When you call \texttt{useScript("jQuery.js");} it does some nifty stuff under the hood.
It pushes the identifer back in an array \texttt{\$SCRIPT}. When the \texttt{view/view.php}
script is called, the \texttt{view/head.php} script should go through the \texttt{\$SCRIPT}
array and echo out the full path to the script with the correct base.\\
\\
\\
\texttt{useScript("jQuery.js");} \\ will produce output that looks a bit like \\
\\
\texttt{<script src = 'http://whube.com/libs/js/jQuery.js' type = 'text/javascript'></script>}
The bit \texttt{http://whube.com/} will of course, be replaced with \texttt{\$SITE\_PREFIX}.
This also allows for the \texttt{useScript()} to rewrite URLs on the fly, and handle conditionals
outside of the view ( content ) code.
